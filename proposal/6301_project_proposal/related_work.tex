While computer power has increased exponentially over the past several years, we still rely on the same interface that was created over 20 years ago, namely a 2D screen with interaction through a keyboard and mouse.
While this is sufficient for every day tasks, it is quickly becoming more and more cumbersome on designers of 3D objects (architects, mechanical engineers, 3D artists, etc) \cite{Dekker199242}.

It has been shown that creating 3D objects within an Immersive Virtual Environment (IVE) has significant benefits over developing on a 2D surface (a monitor)\cite{Kaufmann:Usability}, especially in a classroom setting \cite{Kaufmann:LearningGeometry}.

Previous studies have shown that it is beneficial to break the design in to different groups of tools, such as creation, editing, and grouping
\cite{Butterworth:1992:3DM}.

Also, other research has given evidence that careful attention needs to be placed on the method of interaction \cite{Bowman98interactiontechniques}.
There have been extensive studies on locomotion within IVEs, such as World in Miniature \cite{Pausch:WorldInMiniature}.

Manipulating and creating objects in 3D is a relatively hard task\cite{Mine:MovingObjects}.
One proposal is to use real objects in an Augmented Reality setting
\cite{Jota:2011:CVM:1979742.1979915}.

Another proposal is to use a "3D arrow" to move individual points on a object\cite{5759472}.

Dexterity in IVEs is also a problem.
One group used a pressure sensitive notepad tracked in 3D to simulate a digitizer which allowed manipulation of
objects in the environment\cite{658467}.

There is evidence portraying the lack of useful computer based tools for architecture, especially within architecture education \cite{Dobson:Architecture}.

The ability of an IVE to design and view 3D models dynamically extends well beyond architecture and simple CAD.
Volkswagen has been designing a virtual reality system, RAMSIS, that allows its engineers to study the styling and design of their vehicles before production \cite{Purschke:Cars}.
We believe we can take this one step further: by allowing a user to bring a model in to the world, the user can design an entire world and place objects, such as cars, within it.

Another exciting possibility for the uses of a generic 3D design system is that of Micro-Electro-Mechanical Systems (MEMS) \cite{Zhao:MEMS}.
By extending our system to allow for real-world simulation effects, the user could prototype the device and study its behavior in an intuitive manner.