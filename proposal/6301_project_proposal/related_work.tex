It has been shown that creating 3D objects within an Immersive Virtual Environment (IVE) has significant benefits over developing on a 2D surface (a monitor)\cite{Kaufmann:LearningGeometry,Kaufmann:Usability}.
Previous studies have shows that it is beneficial to break the design in to different groups of tools, such as creation, editing, and grouping \cite{Butterworth:1992:3DM}.
Also, other research has given evidence that careful attention needs to be placed on the method of interaction \cite{Bowman98interactiontechniques}.
There have been extensive studies on locomotion within IVEs, and one that we plan on adapting is World in Miniature \cite{Pausch:WorldInMiniature}.
However, since visual real estate is so limited, rather than create a miniature version of the world in which to navigate, we will simply scale the world to allow the user to navigate much faster than would be possible using real or virtual walking.
