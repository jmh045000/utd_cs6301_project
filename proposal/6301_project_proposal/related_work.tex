It has been shown that creating 3D objects within an Immersive Virtual
Environment (IVE) has significant benefits over developing on a 2D surface (a
monitor)\cite{Kaufmann:Usability}, especially in a 
classroom setting \cite{Kaufmann:LearningGeometry}.

Previous studies have shows that it is beneficial to break the design in to
different groups of tools, such as creation, editing, and grouping
\cite{Butterworth:1992:3DM}.

Also, other research has given evidence that careful attention needs to be
placed on the method of interaction \cite{Bowman98interactiontechniques}.

There have been extensive studies on locomotion within IVEs, and one that we
plan on adapting is World in Miniature \cite{Pausch:WorldInMiniature}.
