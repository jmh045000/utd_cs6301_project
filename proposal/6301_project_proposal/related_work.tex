It has been shown that creating 3D objects within an Immersive Virtual
Environment (IVE) has significant benefits over developing on a 2D surface (a
monitor)\cite{Kaufmann:Usability}, especially in a classroom setting
\cite{Kaufmann:LearningGeometry}.

Previous studies have shown that it is beneficial to break the design in to
different groups of tools, such as creation, editing, and grouping
\cite{Butterworth:1992:3DM}.

Also, other research has given evidence that careful attention needs to be
placed on the method of interaction \cite{Bowman98interactiontechniques}.
There have been extensive studies on locomotion within IVEs, such as World in
Miniature \cite{Pausch:WorldInMiniature}.

Manipulating and creating objects in 3D is a relatively hard task\cite{Mine:MovingObjects}.  One
proposal is to use real objects in a Augmented Reality setting
\cite{Jota:2011:CVM:1979742.1979915}.

Another proposal is to use a "3D arrow" to move individual points on a object\cite{5759472}.

Dexterity in IVEs is also a problem.  One group used a pressure sensitive 
notepad tracked in 3D to simulate a digitizer which allowed manipulation of
objects in the environment\cite{658467}.
