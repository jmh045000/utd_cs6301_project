There are three highly evolved aspects of human beings that characterizes our species as intelligent and set us apart from other animals: our ability to learn, our ability to craft and our ability to communicate.
In fact these abilities are highly inter-related and interlinked to each other in a kind of positive feedback loop.
Our ability to learn helps us assimilate new concepts in terms of previous ones that we learned.
We can compare contrast, generalize or compose the concepts that we have learned to create brand new ideas that can intern be crafted into a useful tool or a wonderful piece of art or a new scientific theory.
We can communicate what we have learned or created to another human being, enabling him to take it from there instead of coming up with these ideas by himself.
It is this positive feed back loop that has been powering the evolution of human beings as the most advanced and intelligent species on earth.
The world builder is a concept of a tool that nitro-boosts this learning-creating-communicating cognitive process. 

The idea behind the world builder concept is simple: Create a world in which the user can craft other worlds.
The simplicity of this idea hides the immense implications of it.
It is relatively easy to see that our ability to craft is magnified with the world builder.
In the beginning of human civilization, a craftsman had to toil for days or weeks to manifest the idea in his head into a tool or structure made of iron or brass.
In today's world of computer aided design and 3d-printers, this work flow is eased into a matter of a few hours.
What virtual reality is offering through the world builder concept is a spontaneous manifestation of what you can imagine in an immersive environment.

There is a growing body of studies that suggests that human beings learn by what is called embodied simulation.
Any new concept is processed in our minds through simulated use of our five senses in the cognitive process of understanding it.
Seymour Papert in an essay titled "The gears of my childhood" recalls how he used to love playing with gear trains as a child.
Later when he was taught multiplication and linear equations he cast this new mathematics concept in terms of his familiar concept of gears by figuring out how many teeth a gear should have to make the equation "run", even imagining himself to be the gear and turning with it.
Now not everyone fell in love with gears as a child.
But some of us definitely remember using our fingers to add two numbers or while moving a variable from the left hand side of an "equal" sign to the right, we remember to change a "plus" in front of it to a "negative" sign or to "pull" it down to the denominator.
As children we were all eager to play with our environment; to explore, try something and see what happens.
We lost track of time when we are focussed in our our play, a concept that psychologists now call "the flow".
It turns out this is the best way to learn!
For one thing, the motivation to explore and learn comes from within us and secondly the assimilation of a concept in terms of actually ``doing'' it (albeit in our minds) is the fastest way to learn.
Does that mean that we give all children a set of gears and expect them to solve linear equations?
Well, no.
Papert's love for gears was something very personal to him.
Gears were Papert's vehicle to transport new ideas to his brain.
He calls it a transitional object.
But each student would need to find his own!
What the World Builder hopes to be is the ultimate lego set to build this transitional object!
No, in fact, a way to create your own lego set, with which you would go on and create that one thing that will be your very own Papert's gear! 

If we think about it, the maximum impact of virtual reality has been in movies and movies are one of the most viable medium of communication to reach the general masses.
Imagine the possibilities if all storytellers irrespective of age had a tool at their disposal with which they can bring to life the very world they paint with words!
Further on, people could share the world that they created that contains the building blocks that they used to understand a concept and thus passing on the core idea to others.
The user of such a world might tinker with this building blocks or remix it to come up with his own personal transitional object. Communication will now take place using embodied and tangible symbols of a concept rather than abstruse words and two dimensional diagrams.
This is the vision for the world builder concept.

Well, these are all tall claims and we don't claim to achieve all of this at the end of our humble class project.
But these are the ideas or the vision that motivates our project.