World builder is a concept of a meta-world that can be used to imagine and create new worlds. It provides the necessary tools in an immersive
environment to aid this process of creation in the most natural way possible. We live in an era of the meta-tool: the computer which like the greek 
mythology of Proteus can assume the form of any tool that can be imagined. But the 2D desktop environment still present some degree of cognitive 
distancing from what we imagine in our minds and what is manifested. The world builder concept aims to reduce this cognitive gap by placing the twenty-
first century descendants of Homo habilis (the human tool user) in an immersive environment where they can craft objects and observe them being formed
at the tip of their fingers.


The concept of the world builder is not new. In 2007, filmmaker Bruce Branit released short filmed titled ``World Builder'' \cite{video:WorldBuilder}
which portrays a man using a futuristic 3D interface to create a small town from scratch starting with simple geometric shapsn progressing through
various levels of detail.  The user goes about creating these with an ease of creating a painting.


The goal of this project is to implement an immersive 3-Dimensional interface that will facilitate the building of a virtual environment from scratch, in
the most natural and easy way for users, at the same time not limiting the user with respect to their creative abilities and imagination. A 
minimum viable implementation of the world builder would include tools that enable the user to construct virtual objects; place prefabricated models 
within the world; manipulate objects in terms of translation, rotation, scaling, or texturing; and a way of navigating this virtual world.


The envisioned applications of the world builder tools are many. Firstly, it can be a most amenable educational tool. There is a growing body of studies 
that suggests that human beings learn by what is called embodied simulation. Any new concept is processed in our minds through simulated use of our five
senses in the cognitive process of understanding it. A good example of this would be the use of lego construction set to learn about 3D geometry and 
spatial co-ordination.  Another contemporary example is the use of minecraft in teaching history, in which the students re-create and explore a 
historical setting in the sandbox game. The world builder can be the ultimate tool to create your own lego set or to portray a narrative which can then
be used in teaching projects. Since both the creator and the user can be immersed in a virtual environment which can simulate both real world physics 
and any imagined unreal (read magical) dynamics of objects, such a platform can be very versatile. 


Secondly, the world builder is relevant in the computer aided design field. Immersive CAD applications have been already developed but not for
building an entire world. The world builder application would be able to create an entire city instead of designing one single building in an immersive 
environment. The world builder would not only be able to create single entities but would also enable the study of the dynamics of a system that contains
a multitude of objects. Finally, the movie and the entertainment industry has historically been the largest consumer of virtual reality 
applications. The world builder concept has immense utility for laying out a scene either for storyboarding or to be rendered in an animated movie.


This project aims to identify functionalities supported by existing CAD systems, conceptualize functionalities specifically for world building, design
a 3D user interface that will support these old and new functionalities which will provide an interaction that feels intuitive and natural to the user
and finally, and create a minimum viable product that contains a subset of these functionalities as a proof of concept.
Section 2 outlines previous work done by other groups.  
Section 3 outlines the design of the end result.
Section 4 details the plan of action for creating the prototype application.
